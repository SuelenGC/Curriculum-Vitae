\documentclass[a4paper, oneside, final]{scrartcl}

\usepackage[utf8]{inputenc}
\usepackage[brazil]{babel}
\usepackage{amssymb}

\usepackage{soul}
\usepackage{scrpage2}
\usepackage{titlesec}
\usepackage{marvosym}
\usepackage{tabularx}
\usepackage{textcomp}
\newcommand{\vspc}{\vspace{0.15cm}} %espaçamento vertical
\newcommand{\vspcitem}{\vspace{0.1cm}} %espaçamento vertical

\usepackage[hmargin=1.0cm,vmargin=1.5cm,noheadfoot]{geometry}

\titleformat{\section}{\large\scshape\raggedright}{}{0em}{}[\titlerule]
\pagestyle{scrheadings}
\renewcommand{\headfont}{\normalfont\rmfamily\scshape}

% add the symbols for email and phone contact data
\cofoot{
{\small\Letter} {\large suelengcarvalho@gmail.com} \ {\small\Telefon} {\small +55 (11) 99336-7069}
} 


\begin{document}

\begin{center}
\textsc{\Huge{Suelen Goularte Carvalho}} \vspc\\
{\small Brasileira, +55 (11) 99336-7069, suelengcarvalho@gmail.com} 

%\textsc{\huge{\so{Suelen Goularte Carvalho}}}\\ \ \\
%{\small \tt Preten��o Salarial: R\$ 10.000,00 CLT Full}

\section{Objetivo}
	{\bf Atuar como arquiteta de software, líder técnico ou especialista.}

%\section{Personal Information}

%\begin{tabularx}{.97\linewidth}{p{3.0cm}p{4.0cm}p{3.0cm}X}
%Phone:           & +55 (11) 9336-7069 & Email:        & suelengcarvalho@gmail.com \\
%Date of Birth: & 26/02/1987     & Marital Status: & Single\\
%\                   & \              & Linkedin:     & %http://www.linkedin.com/in/suelengc 
%\end{tabularx}

\section{Resumo}
\begin{tabularx}{1\linewidth}{X}
	$\star$ Arquiteta de software Web e Mobile com mais de 8 anos de experiência em programação e com grande entusiasmo por software e novas tecnologias. \vspc\\
	
	$\star$ Excelentes conhecimentos em diferentes linguagens de programação e tecnologias, incluindo Java, .Net, Scala, C/C++, Android, tecnologias web e SQL. \vspc\\
\end{tabularx}

\section{Idiomas}
\begin{tabularx}{1\linewidth}{p{2cm}X}
Português      	& Nativo\\
Inglês      	& Avançado\\
\end{tabularx}


\section{Formação Acadêmica}
\begin{tabularx}{1\linewidth}{p{2cm}X}
2010$-$2012 & {\bf Pós-Graduação em Tecnologia da Informação}\\
            & Instituto Tecnológico de Aeronáutica, ITA\\
            & Título: Padrões Para Implantar Métodos Ágeis\\            
            & Área: Gestão Estratégica de Projetos \vspc\\
\end{tabularx}
\begin{tabularx}{1\linewidth}{p{2cm}X}
2005$-$2008 & {\bf Graduação em Análise e Desenvolvimento de Sistemas}\\
            & Faculdade de Tecnologia Diamantes, FATED \vspc\\
\end{tabularx}
\begin{tabularx}{1\linewidth}{p{2cm}X}
2003$-$2003 & {\bf Técnico em Processamento de Dados}\\
            & Escola Profissional Nossa Senhora de Fátima, SENAI\\
\end{tabularx}


\section{Formação Complementar}
\begin{tabularx}{1\linewidth}{p{2cm}X}
2013       & Jenkins: Automação de deploy e Integração Contínua (12h) - Caelum \vspcitem\\
\end{tabularx}

\begin{tabularx}{1\linewidth}{p{2cm}X}
2013       & Maven: Gerenciando de dependências (12h) - Caelum \vspcitem\\
\end{tabularx}

\begin{tabularx}{1\linewidth}{p{2cm}X}
2013       & Testes de Unidade e TDD (16h) - Caelum \vspcitem\\
\end{tabularx}

\begin{tabularx}{1\linewidth}{p{2cm}X}
2013       & Java Web (40h) - Caelum \vspcitem\\
\end{tabularx}

\begin{tabularx}{1\linewidth}{p{2cm}X}
2012       & Gerenciamento ágil de projetos de Software com Scrum (20h) - Caelum \vspcitem\\
\end{tabularx}

\begin{tabularx}{1\linewidth}{p{2cm}X}
2012       & Otimizando o Fluxo de Trabalho - Virada Ágil (8h) - Thought Works \vspcitem\\
\end{tabularx}

\begin{tabularx}{1\linewidth}{p{2cm}X}
2012       & Google Maps Engine (20h) - Google \vspcitem\\
\end{tabularx}

\begin{tabularx}{1\linewidth}{p{2cm}X}
2012       & Google Maps API (20h) - Google \vspcitem\\
\end{tabularx}

\begin{tabularx}{1\linewidth}{p{2cm}X}
2012       & Google Android Hands On (8h) - Tempo Real Eventos \vspcitem\\
\end{tabularx}

\begin{tabularx}{1\linewidth}{p{2cm}X}
2010       & Academia Java (128h) - Globalcode \vspcitem\\
\end{tabularx}

\begin{tabularx}{1\linewidth}{p{2cm}X}
2010       & Álgebra Linear (120h) - Universidade de São Paulo, USP \vspcitem\\
\end{tabularx}

\begin{tabularx}{1\linewidth}{p{2cm}X}
2009       & Tópicos de Programação (60h) - Universidade de São Paulo, USP \vspcitem\\
\end{tabularx}

\begin{tabularx}{1\linewidth}{p{2cm}X}
2008       & Academia .Net (VB, ASP, C\# e ADO) (144h) - Ká Solutions 
\end{tabularx}


\section{Experiência Profissional}
\begin{tabularx}{1\linewidth}{X}
{\bf MapLink $\star$ Arquiteto de Software \hfill Ago/2012 $-$ Atual} \\
Líder técnica do time de projetos corporativos composto por 8 pessoas. Atuando diretamente na definição de arquitetura e design de web services e bibliotecas mobile Android e iOS do MapLink. Principal responsável pela evolução das bibliotecas mobile e conteúdo do site dev.maplink.com.br/mobile-api. Atuando diretamente no desenvolvimento da biblioteca Android. Aplicação de integração contínua com Jenkins e Maven, execução automatizada de testes e code review. Estas bibliotecas visam facilitar o uso de serviços web do MapLink em aplicativos móveis. \vspc\\
\end{tabularx}

\begin{tabularx}{1\linewidth}{X}
{\bf Liberty Seguros $\star$ Líder de Projetos Web \hfill Abr/2011 $-$ Jul/2012} \\
Líder de projetos responsável pela evolução do Portal do Segurado, principal software disponibilizado aos segurados Liberty. Atuando com times terceirizados de até 6 pessoas. Foram utilizados conceitos da cultura ágil para a gestão dos projetos. Ao longo de pouco mais de 1 ano foram entregues dois projetos de grande peso estratégico para a companhia. Os desafios incluíam questões de performance, escalabilidade, segurança da informação e agilidade na entrega dos projetos.\vspc\\
\end{tabularx}

\begin{tabularx}{1\linewidth}{X}
{\bf Serasa Experian $\star$  Analista de Sistemas \hfill Out/2010 $-$ Abr/2011} \\
Atuando diretamente na implantação de práticas ágeis tais como Lean, Kanban e Scrum no time de sustentação composto por 5 pessoas. Criado parcerias com áreas próximas ao nosso dia-a-dia. Forte atuação na evolução de softwares desenvolvidos em JavaEE e C\#.Net \vspc\\
\end{tabularx}

\begin{tabularx}{1\linewidth}{X}
{\bf Liberty Seguros $\star$ Analista de Sistemas Sênior \hfill Nov/2006 $-$ Out/2010} \\
Responsável direta desde o desenvolvimento até a implantação de evoluções e correções no sistema Affinity Web. Arquitetura baseada no conceito de múltiplas camadas e SOA utilizando o protocolo Soap e XML. Desenvolvido usando a plataforma .Net.\vspc\\
\end{tabularx}

\begin{tabularx}{1\linewidth}{X}
{\bf Emphasys Consultoria $\star$ Analista Programador Pleno  \hfill Set/2006 $-$ Nov/2006} \\
Atuação no desenvolvimento com Visual Basic e Delphi. \vspc\\
\end{tabularx}

\begin{tabularx}{1\linewidth}{X}
{\bf Signa Logística do E-Business $\star$ Programador Pleno  \hfill Abr/2006 $-$ Set/2006} \\
Desenvolvimento de sistemas utilizando multiplas tecnologias $-$ ASP, VBScript, Visual Basic, HTML, XML, CSS e JavaScript. Desenvolvimento e manutenção de stored procedures, triggers em sistemas Oracle e Microsoft SQL Server. \vspc\\
\end{tabularx}

\section{Experiência Acadêmica}
\begin{tabularx}{1\linewidth}{X}
{\bf Caelum Ensino e Inovação $\star$ Instrutor  \hfill Out/2012 $-$ Atual} \\
Treinamentos dos cursos de Java para Web, Gerenciamento Ágil de Projetos de Software com Scrum, Desenvolvimento Móvel com Google Android e Técnicas de Desenvolvimento Android avançado. \vspc\\
\end{tabularx}

\begin{tabularx}{1\linewidth}{X}
{\bf Universidade Bandeirantes (UNIBAN) $\star$ Professor \hfill Abr/2009 $-$ Ago/2009} \\
Professor da disciplina de graduação Desenvolvimento Web. \vspc\\
\end{tabularx}

\begin{tabularx}{1\linewidth}{X}
{\bf Faculdade de Tecnologia Diamantes (FATED) $\star$ Professor \hfill Jun/2008 $-$ Jun/2009} \\
Professor das disciplinas de graduação Desenvolvimento Web e Práticas de Orientação a Objeto com Java. \vspc\\
\end{tabularx}

\begin{tabularx}{1\linewidth}{X}
{\bf Faculdade Paulista de Artes (FPA) $\star$ Professor  \hfill Mai/2007 $-$ Jul/2007} \\
Professor da disciplina do curso extracurricular Desenvolvimento Web com PHP. \vspc\\
\end{tabularx}


\section{Participações em Eventos (Como Palestrante)}
\begin{tabularx}{1\linewidth}{p{2.0cm}X}
08/2013    & QConSP - Nativo ou Cross-Plataform: Caminhos e Alternativas \vspcitem\\
\end{tabularx}
\begin{tabularx}{1\linewidth}{p{2.0cm}X}
07/2013    & 5º GDG Android Meetup - sobre O bê-a-bá de Widgets \vspcitem\\
\end{tabularx}
\begin{tabularx}{1\linewidth}{p{2.0cm}X}
07/2013    & 3º GDG Android Meetup - Usando o Poder da API Fragments \vspcitem\\
\end{tabularx}
\begin{tabularx}{1\linewidth}{p{2.0cm}X}
04/2013    & DevFestW - Primeiros Passos com Android \vspcitem\\
\end{tabularx}
\begin{tabularx}{1\linewidth}{p{2.0cm}X}
11/2012    & DevFest - Esquartejando sua Activity com Fragments \vspcitem\\
\end{tabularx}
\begin{tabularx}{1\linewidth}{p{2.0cm}X}
09/2012    & Agile Brazil - Padrões Para Implantar Métodos Ágeis \vspcitem\\
\end{tabularx}
\begin{tabularx}{1\linewidth}{p{2.0cm}X}
08/2012    & QConSP - SOLID em 5 minutos \vspcitem\\
\end{tabularx}
\begin{tabularx}{1\linewidth}{p{2.0cm}X}
07/2012    & TDC - The Developer's Conference - 7 Padrões Para Implantar Métodos Ágeis \vspcitem\\
\end{tabularx}
\begin{tabularx}{1\linewidth}{p{2.0cm}X}
07/2012    & TDC - The Developer's Conference - Deixando sua Interface mais Bonita com Shapes \vspcitem\\
\end{tabularx}
\begin{tabularx}{1\linewidth}{p{2.0cm}X}
09/2011    & MiniPLoP Brasil - Artigo Padrões Para Implantar Métodos Ágeis 
\end{tabularx}

\section{Conhecimentos}
\begin{tabularx}{1\linewidth}{p{3.5cm}X}
Práticas Ágeis: 	& Scrum, Lean, Kanban, XP, TDD e DDD.\\
Práticas:  			& Design Patterns, Orientação a Objetos.\\
Linguagens:       	& Java, JSP, Servlet, JSTL, EL, C\#.Net, ASP.Net, VB.Net, ASP, PHP, C/C++.\\
Tecnologias:    	& HTML, CSS, XML, XSL, DTD, Javascript, Json, Ajax, LaTeX e SQL.\\
Mobile:          	& Android, ActionBarSherlock, KSoap2, Fragments, Robotium, HttpSync, AsyncTask, SQLite.\\
Frameworks:      	& .Net até 4.5, Spring, Hibernate, JQuery.\\
Banco de Dados:     & MySQL, Oracle and Microsoft SQL Server \\
Servidores:         & Websphere, IIS, Tomcat e Apache.\\
IDE's:            	& Eclipse, MS Visual Studio, Netbeans.\\
Ferramentas: 		& Maven, Jenkins, Hudson.\\
Controle de Versão: & Github, Bitbucket, Source Safe, TFS.\\
Aplicativos:    	& Photoshop, Ilustrator and Dreamweaver.\\
\end{tabularx}

\end{center}

\end{document}