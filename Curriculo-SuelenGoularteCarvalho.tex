\documentclass[a4paper, oneside, final]{scrartcl}

\makeatletter
	\DeclareOldFontCommand{\rm}{\normalfont\rmfamily}{\mathrm}
	\DeclareOldFontCommand{\sf}{\normalfont\sffamily}{\mathsf}
	\DeclareOldFontCommand{\tt}{\normalfont\ttfamily}{\mathtt}
	\DeclareOldFontCommand{\bf}{\normalfont\bfseries}{\mathbf}
	\DeclareOldFontCommand{\it}{\normalfont\itshape}{\mathit}
	\DeclareOldFontCommand{\sl}{\normalfont\slshape}{\@nomath\sl}
	\DeclareOldFontCommand{\sc}{\normalfont\scshape}{\@nomath\sc}
\makeatother

\usepackage[utf8]{inputenc}
\usepackage[brazil]{babel}
\usepackage{amssymb}

\usepackage{soul}
\usepackage{scrpage2}
\usepackage{titlesec}
\usepackage{marvosym}
\usepackage{tabularx}
\usepackage{textcomp}
\usepackage[hmargin=1.0cm,vmargin=1.5cm,noheadfoot]{geometry}

\usepackage{hyperref}
\newcommand{\vspc}{\vspace{0.15cm}} %espaçamento vertical
\newcommand{\vspcitem}{\vspace{0.1cm}} %espaçamento vertical

\titleformat{\section}{\large\scshape\raggedright}{}{0em}{}[\titlerule]
\pagestyle{scrheadings}
\renewcommand{\headfont}{\normalfont\rmfamily\scshape}

% add the symbols for email and phone contact data
\cofoot{
  {\small\Letter} {\small suelengcarvalho@gmail.com} \ {\small\Telefon} {\small +55 (11) 95020-0060} 
} 


\begin{document}

\begin{center}
\textsc{\Huge{Suelen Goularte Carvalho - 30 anos}} \vspc\\
{\small Brasileira, +55 (11) 95020-0060, suelengcarvalho@gmail.com} 

%\textsc{\huge{\so{Suelen Goularte Carvalho}}}\\ \ \\
%{\small \tt Preten��o Salarial: R\$ 10.000,00 CLT Full}

%\section{Objetivo}
%	{\bf Atuar com Liderança Técnica e Desenvolvimento de Software. }

%\section{Personal Information}

%\begin{tabularx}{.97\linewidth}{p{3.0cm}p{4.0cm}p{3.0cm}X}
%Phone:           & +55 (11) 9336-7069 & Email:        & suelengcarvalho@gmail.com \\
%Date of Birth: & 26/02/1987     & Marital Status: & Single\\
%\                   & \              & Linkedin:     & %http://www.linkedin.com/in/suelengc , suelengc.com.br
%\end{tabularx}

\section{Resumo}
\begin{tabularx}{1\linewidth}{X}

  % Atualmente busca colaborar de forma estratégica do ponto de vista de tecnologia. \vspc\\

  % $\star$ Durante sua carreira, Suelen passou por todas as etapas do desenvolvimento de software. Atuou desde posições especialistas como Desenvolvedora e Arquiteta de Software, até posições de liderança como Líder Técnica e Agile Coach, lidando com times de até 40 pessoas. Atuou em times com foco no desenvolvimento de produtos e outros focados em incidentes do dia a dia. Durante esses 12 anos ganhou um amplo e sólido conhecimento sobre processos e melhores práticas de engenharia, desenvolvimento de produtos e gestão de pessoas que atuam nesses times. \vspc\\

  % $\star$ Apaixonada por conhecimento e sempre disposta a melhorar, Suelen nunca parou de estudar. Cursou técnico em Processamento de Dados, se graduou em Análise e Desenvolvimento de Sistemas, pós-graduou em Gestão Estratégica de Projetos e atualmente está no último ano de Mestrado em Ciência da Computação. \vspc\\

  % $\star$ Atualmente colabora nas empresas de forma estratégica auxiliando na integração entre o modelo de negócio e tecnologia focando na entrega de valor. Desta maneira, consegue sugerir o uso de tecnologias, processos e práticas mais adequados para evolução dos produtos, sistemas e times, sem se afastar do desenvolvimento de software. \vspc\\

  % $\star$ Além de sua vida profissional, Suelen adora praticar esportes, viajar e estar com sua família. \vspc\\


  $\star$ Atua há 12 anos com desenvolvimento de software web e mobile e 7 anos com liderança. Durante estes anos passou por diversas tecnologias como Java, Android, Ruby, Go, Javascript, HTML/CSS, SQL, Kotlin, Cloud AWS (RDS, ECS, e outros), Docker, ferramentas de CI (CircleCI, Jenkins e BuddyBuild), etc.  \vspc \\

  $\star$ Atuou em grandes empresas e startups o que contribuiu para adquirir sólidas experiências em processos de engenharia e melhores práticas usando abordagens como Lean Startup, Kanban, Scrum, TDD, XP e outros. \vspc \\

  $\star$ Possui sólida formação acadêmica tendo cursado técnico em Processamento de Dados (SENAI), graduação em Análise e Desenvolvimento de Sistemas (FATED), especialização em Gestão de Projetos (ITA) mestrado em Ciência da Computação (IME-USP). \vspc \\

  $\star$ Participante ativa na comunidade, já apresentou mais de 30 palestras voluntárias nos últimos 6 anos e atuou na organização de alguns grandes eventos como Agile Brazil e Android Meetup enquanto fez parte do GDG-SP.
\end{tabularx}

  % $\star$ Grande experiência com arquitetura de software web e mobile com mais de 12 anos com programação e liderança. Muito entusiasmada por software e novas tecnologias. \vspc\\
  
  % $\star$ Ótimas skills interpessoais, organizada, persistente e auto-motivada. Em constante busca por conhecimento, melhores resultados e desafios. \vspc\\
  
  % $\star$ Profundo conhecimento sobre a Cultura Ágil e Métodos Ágeis tais como Scrum, Kanban, Lean, XP e TDD. \vspc\\
  
  % $\star$ Excelente conhecimento em diferentes linguagens de programação e tecnologias incluindo Java, .Net, Scala, C/C++, Android, tecnologias web e SQL. \vspc\\
  
  % $\star$ Participante ativa na comunidade por meio de eventos de tecnologia como palestrante, ouvinte e organizadora, grupos de discussão online e textos técnicos em revistas e blog pessoal. \vspc\\ 

\section{Idiomas}
\begin{tabularx}{1\linewidth}{p{6cm}X}
$\star$ Inglês: Fluência Básica & $\star$ Português: Nativo\\
\end{tabularx}


\section{Formação Acadêmica}
\begin{tabularx}{1\linewidth}{p{2cm}X}
2014$-$2017 & {\bf Mestrado em Ciencia da Computação}\\
            & Instituto de Matemática e Estatística da Universidade de São Paulo, IME-USP\\
            & Título: Maus cheiros de código no \textit{front-end} Android\\
            & Área: Computação Móvel \vspc\\
\end{tabularx}
\begin{tabularx}{1\linewidth}{p{2cm}X}
% 2010$-$2012 & {\bf Pós-Graduação em Tecnologia da Informação}\\
2010$-$2012 & {\bf Pós-Graduação em Gestão de Projetos}\\
            & Instituto Tecnológico de Aeronáutica, ITA\\
            & Título: Padrões Para Implantar Métodos Ágeis\\            
            & Área: Gestão Estratégica de Projetos \vspc\\
\end{tabularx}
\begin{tabularx}{1\linewidth}{p{2cm}X}
2005$-$2008 & {\bf Graduação em Análise e Desenvolvimento de Sistemas}\\
            & Faculdade de Tecnologia Diamantes, FATED \vspc\\
\end{tabularx}
\begin{tabularx}{1\linewidth}{p{2cm}X}
2003$-$2003 & {\bf Técnico em Processamento de Dados}\\
            & Escola Profissional Nossa Senhora de Fátima, SENAI\\
\end{tabularx}


\section{Experiência Profissional}
\begin{tabularx}{1\linewidth}{X}
{\bf Moip $\star$ Agile Coach e Líder Técnica \hfill Sep/2016 $-$ Atual} \\
Como Agile Coach sou responsável pela disseminação, implementação e evolução do uso de métodos e práticas ágeis dentro do time de Produtos \& Engenharia, que na época era composto por 45 pessoas, porém, não me limitando a eles. Tenho feito isso através de workshops sobre agilidade, coach e mentoria constantes com os times, estando sempre a disposição. Muitas das práticas que usamos no dia a dia vem do Kanban, Lean e Scrum. Como Líder Técnica sou responsável pela evolução técnica e gestão dos membros do time. Desta forma busco automatizar processos que tornam o time mais ágil, removo impedimentos, incentivo a disseminação de conhecimento e multidisciplinaridade, realizo 1:1 semanais e feedbacks semestrais com os desenvolvedores, promovo o uso das melhores práticas de desenvolvimento. No dia a dia atuo também como desenvolvedora implementando features, realizando code reviews, promovendo o estudo de novas tecnologias, dentre outras atividades. Algumas das ferramentas e tecnologias que usamos no dia a dia são: Cloud AWS, Docker, Jira/Confluence, Slack, Git/Github, CircleCI, Hockeyapp, Fabric, Android, Java, Kotlin, Ruby, Swift (iOS), dentre muitas outras. \vspc\\
\end{tabularx}

\begin{tabularx}{1\linewidth}{X}
{\bf Tripda $\star$ Líder Mobile \hfill Abr/2015 $-$ Mar/2016} \\
Como líder do time de mobile fui responsável pela boa comunicação com a equipe do Produtos e evoluir práticas ágeis. Como desenvolvedora trabalhei no desenvolvimento de aplicativos usando a ferramenta Xamarin e C\#. Nosso processo de desenvolvimento incluia testes automatizados, ambiente de integração contínua gamificado, feature branchs, code review e refatorações constantes. Utilizávamos o Xamarin Insights e Crittercism (agora Apteligent) como ferramenta para relatório de falhas, Localytics e Google Analytics como ferramentas de análise, AppBot para monitorar a classificação do usuário integrado com Slack e o HockeyApp para distribuição interna de versões de teste. \vspc\\
\end{tabularx}

\begin{tabularx}{1\linewidth}{X}
{\bf VivaReal $\star$ Gerente de Engenharia \hfill Ago/2014 $-$ Mar/2015} \\
Fui responsável pelo gerenciamento de quatro times de engenheiros realizando 1-1s, feedback, contratação, demissão, promoções e promovendo o uso de boas práticas e processos de desenvolvimento. Esses times usavam diferentes tecnologias como Python, Java, Android, iOS, AngularJS, HTML5, CSS3, MongoDB, entre outros. Todos nossos aplicativos estavam sob a infraestrutura AWS. \vspc\\
\end{tabularx}

\begin{tabularx}{1\linewidth}{X}
{\bf Caelum $\star$ Desenvolvedora \hfill Nov/2014 $-$ Ago/2014} \\
Desenvolvendo e melhorando projetos utilizando TDD e outros métodos ágeis. Ministrando cursos de Java para Web, Gerenciamento Ágil de projetos de Software com Scrum, Desenvolvimento móvel com Google Android, Técnicas de Desenvolvimento Android Avançado e outros. \vspc\\
\end{tabularx}

\begin{tabularx}{1\linewidth}{X}
{\bf MapLink $\star$ Arquiteta de Software \hfill Ago/2012 $-$ Nov/2013} \\
Líder técnica de um time composto por 8 pessoas. Atuando diretamente na definição de arquitetura e design de web services e API mobile Android e iOS do MapLink. Principal responsável pela evolução das APIs mobile, que visam facilitar o uso de serviços web do MapLink em aplicativos móveis, e conteúdo do site dev.maplink.com.br/mobile-api. Atuando também no desenvolvimento da API Android. Introduzido conceitos de integração contínua com Jenkins e Maven, análise estática de código, execução automatizada de testes e code review através do Github. \vspc\\
\end{tabularx}

\begin{tabularx}{1\linewidth}{X}
{\bf Liberty Seguros $\star$ Líder de Projetos Web \hfill Abr/2011 $-$ Jul/2012} \\
Líder de projetos responsável pela evolução do Portal do Segurado, principal software disponibilizado aos segurados Liberty. Atuando com times terceirizados de até 6 pessoas. Foram utilizados conceitos da cultura ágil para a gestão dos projetos. Ao longo de pouco mais de 1 ano foram entregues dois projetos de grande peso estratégico para a companhia. Os desafios incluíram questões de performance, escalabilidade, segurança da informação e agilidade na entrega dos projetos. As tecnologias utilizadas foram .Net, tecnologias web e Team Foundation Server.\vspc\\
\end{tabularx}

\begin{tabularx}{1\linewidth}{X}
{\bf Serasa Experian $\star$  Analista de Sistemas \hfill Out/2010 $-$ Abr/2011} \\
Atuando diretamente na implantação de práticas ágeis tais como Lean, Kanban e Scrum no time de sustentação composto por 5 pessoas. Criado parcerias com áreas próximas ao nosso dia-a-dia. Forte atuação na evolução de softwares desenvolvidos em JavaEE e C\#.Net \vspc\\
\end{tabularx}

\begin{tabularx}{1\linewidth}{X}
{\bf Liberty Seguros $\star$ Analista de Sistemas Sênior \hfill Nov/2006 $-$ Out/2010} \\
Responsável direta desde o desenvolvimento até a implantação de evoluções e correções no sistema Affinity Web. Arquitetura baseada no conceito de múltiplas camadas e SOA utilizando o protocolo Soap e XML. Desenvolvido utilizando a plataforma .Net e o Team Foundation Server.\vspc\\
\end{tabularx}

\begin{tabularx}{1\linewidth}{X}
{\bf Emphasys $\star$ Analista Programadora Pleno  \hfill Set/2006 $-$ Nov/2006} \\
Atuação no desenvolvimento com Visual Basic e Delphi. \vspc\\
\end{tabularx}

\begin{tabularx}{1\linewidth}{X}
{\bf Signa $\star$ Programadora Pleno  \hfill Abr/2006 $-$ Set/2006} \\
Desenvolvimento de sistemas utilizando multiplas tecnologias $-$ ASP, VBScript, Visual Basic, HTML, XML, CSS e JavaScript. Desenvolvimento e manutenção de stored procedures, triggers em sistemas Oracle e Microsoft SQL Server. \vspc\\
\end{tabularx}

\section{Experiência Acadêmica}
\begin{tabularx}{1\linewidth}{X}
{\bf Caelum $\star$ Instrutora \hfill Out/2012 $-$ Atual} \\
Treinamentos dos cursos de Java para Web, Gerenciamento Ágil de Projetos de Software com Scrum, Desenvolvimento Móvel com Google Android e Técnicas de Desenvolvimento Android Avançado. \vspc\\
\end{tabularx}

\begin{tabularx}{1\linewidth}{X}
{\bf Universidade Bandeirantes (UNIBAN) $\star$ Professora \hfill Abr/2009 $-$ Ago/2009} \\
Professor da disciplina de graduação Desenvolvimento Web. \vspc\\
\end{tabularx}

\begin{tabularx}{1\linewidth}{X}
{\bf Faculdade de Tecnologia Diamantes (FATED) $\star$ Professora \hfill Jun/2008 $-$ Jun/2009} \\
Professor das disciplinas de graduação Desenvolvimento Web e Práticas de Orientação a Objeto com Java. \vspc\\
\end{tabularx}

\begin{tabularx}{1\linewidth}{X}
{\bf Faculdade Paulista de Artes (FPA) $\star$ Professora  \hfill Mai/2007 $-$ Jul/2007} \\
Professor da disciplina do curso extracurricular Desenvolvimento Web com PHP. \vspc\\
\end{tabularx}

\section{Formação Complementar}
\begin{tabularx}{1\linewidth}{p{2cm}X}
2017       &  Neurociência e Comunicação Não Violenta (6h) - Casa do Saber \vspcitem\\
\end{tabularx}
\begin{tabularx}{1\linewidth}{p{2cm}X}
2016       &  Desenvolvimento de Novos Líderes (16h) - Paideia Educação \vspcitem\\
\end{tabularx}
% \begin{tabularx}{1\linewidth}{p{2cm}X}
% 2015       &  As Melhores Ferramentas e Práticas para o Gerenciamento de Pessoas (16h) - Paideia Educação \vspcitem\\
% \end{tabularx}

\begin{tabularx}{1\linewidth}{p{2cm}X}
2015       & Desenvolvimento Mobile com iOS (40h) - Caelum \vspcitem\\
\end{tabularx}

\begin{tabularx}{1\linewidth}{p{2cm}X}
2015       & Desenvolvimento Web com HTML, CSS e JavaScript (40h) - Caelum \vspcitem\\
\end{tabularx}

\begin{tabularx}{1\linewidth}{p{2cm}X}
2013       & Jenkins: Automação de deploy e Integração Contínua (12h) - Caelum \vspcitem\\
\end{tabularx}

\begin{tabularx}{1\linewidth}{p{2cm}X}
2013       & Maven: Gerenciando de dependências (12h) - Caelum \vspcitem\\
\end{tabularx}

\begin{tabularx}{1\linewidth}{p{2cm}X}
2013       & Testes de Unidade e TDD (16h) - Caelum \vspcitem\\
\end{tabularx}

\begin{tabularx}{1\linewidth}{p{2cm}X}
2013       & Java Web (40h) - Caelum \vspcitem\\
\end{tabularx}

\begin{tabularx}{1\linewidth}{p{2cm}X}
2012       & Gerenciamento ágil de projetos de Software com Scrum (20h) - Caelum \vspcitem\\
\end{tabularx}

\begin{tabularx}{1\linewidth}{p{2cm}X}
2012       & Otimizando o Fluxo de Trabalho - Virada Ágil (8h) - Thought Works \vspcitem\\
\end{tabularx}

\begin{tabularx}{1\linewidth}{p{2cm}X}
2012       & Google Maps Engine (20h) - Google \vspcitem\\
\end{tabularx}

\begin{tabularx}{1\linewidth}{p{2cm}X}
2012       & Google Maps API (20h) - Google \vspcitem\\
\end{tabularx}

\begin{tabularx}{1\linewidth}{p{2cm}X}
2012       & Google Android Hands On (8h) - Tempo Real Eventos \vspcitem\\
\end{tabularx}

\begin{tabularx}{1\linewidth}{p{2cm}X}
2010       & Academia Java (128h) - Globalcode \vspcitem\\
\end{tabularx}

\begin{tabularx}{1\linewidth}{p{2cm}X}
2010       & Álgebra Linear (120h) - Universidade de São Paulo, USP \vspcitem\\
\end{tabularx}

\begin{tabularx}{1\linewidth}{p{2cm}X}
2009       & Tópicos de Programação (60h) - Universidade de São Paulo, USP \vspcitem\\
\end{tabularx}

\begin{tabularx}{1\linewidth}{p{2cm}X}
2008       & Academia .Net (VB, ASP, C\# e ADO) (144h) - Ká Solutions 
\end{tabularx}

\section{Participações em Eventos(Como Palestrante) / Publicações}
\begin{tabularx}{1\linewidth}{p{2.0cm}X}
09/2017    & DevFest Maceió - Introdução a Android Instant Apps \vspcitem\\
\end{tabularx}
\begin{tabularx}{1\linewidth}{p{2.0cm}X}
07/2017    & Android Dev Conference - Introdução a Kotlin \vspcitem\\
\end{tabularx}
\begin{tabularx}{1\linewidth}{p{2.0cm}X}
10/2015    & DevFest Sudeste - O sucesso do seu app está nos detalhes \vspcitem\\
\end{tabularx}
\begin{tabularx}{1\linewidth}{p{2.0cm}X}
09/2015    & DevFest Goias - As últimas novidades que todo desenvolvedor deve saber sobre Android \vspcitem\\
\end{tabularx}
\begin{tabularx}{1\linewidth}{p{2.0cm}X}
05/2014    & Mobile Conf RJ - Sua carreira e o que o desenvolvimento mobile tem haver com isso! \vspcitem\\
\end{tabularx}
\begin{tabularx}{1\linewidth}{p{2.0cm}X}
05/2014    & Conexão Java - Sua primeira app Android \vspcitem\\
\end{tabularx}
\begin{tabularx}{1\linewidth}{p{2.0cm}X}
04/2014    & Seven Masters Android - 7 Dúvidas Gerais \vspcitem\\
\end{tabularx}
\begin{tabularx}{1\linewidth}{p{2.0cm}X}
08/2013    & Revista iMasters - Android Studio: Vantagens e Desvantagens com relação ao Eclipse \vspcitem\\
\end{tabularx}
\begin{tabularx}{1\linewidth}{p{2.0cm}X}
08/2013    & QConSP - Nativo ou Cross-Plataform: Caminhos e Alternativas \vspcitem\\
\end{tabularx}
\begin{tabularx}{1\linewidth}{p{2.0cm}X}
07/2013    & 5º GDG Android Meetup - sobre O bê-a-bá de Widgets \vspcitem\\
\end{tabularx}
\begin{tabularx}{1\linewidth}{p{2.0cm}X}
07/2013    & 3º GDG Android Meetup - Usando o Poder da API Fragments \vspcitem\\
\end{tabularx}
\begin{tabularx}{1\linewidth}{p{2.0cm}X}
04/2013    & DevFestW - Primeiros Passos com Android \vspcitem\\
\end{tabularx}
\begin{tabularx}{1\linewidth}{p{2.0cm}X}
11/2012    & DevFest - Esquartejando sua Activity com Fragments \vspcitem\\
\end{tabularx}
\begin{tabularx}{1\linewidth}{p{2.0cm}X}
09/2012    & Agile Brazil - Padrões Para Implantar Métodos Ágeis \vspcitem\\
\end{tabularx}
\begin{tabularx}{1\linewidth}{p{2.0cm}X}
08/2012    & QConSP - SOLID em 5 minutos \vspcitem\\
\end{tabularx}
\begin{tabularx}{1\linewidth}{p{2.0cm}X}
07/2012    & TDC - The Developer's Conference - 7 Padrões Para Implantar Métodos Ágeis \vspcitem\\
\end{tabularx}
\begin{tabularx}{1\linewidth}{p{2.0cm}X}
07/2012    & TDC - The Developer's Conference - Deixando sua Interface mais Bonita com Shapes \vspcitem\\
\end{tabularx}
\begin{tabularx}{1\linewidth}{p{2.0cm}X}
09/2011    & MiniPLoP Brasil - Artigo Padrões Para Implantar Métodos Ágeis 
\end{tabularx}

\section{Trabalhos Voluntários (Hobbie)}
\begin{tabularx}{1\linewidth}{X}
{\bf Líder Organizador do GDG-SP Android Meetup \hfill Abr/2013 - 2015} \\
Desde de Abril de 2013 faço parte do GDG-SP (Google Developers Group - São Paulo) como organizadora líder do Android Meetup. O Android Meetup é um evento focado no desenvolvimento com a plataforma Android. Este encontro é realizado pelo grupo GDG-SP que organiza voluntariamente diversos eventos sobre tecnologias do Google. \vspc\\
\end{tabularx}

\begin{tabularx}{1\linewidth}{X}
{\bf Organizador do Agile Brazil \hfill 2012 - 2014} \\
O Agile Brazil é o maior evento sobre agilidade que temos no Brasil. Em 2012, fui convidada à participar do grupo que o organiza voluntariamente. Desde então, sempre que posso colaboro com a organização do evento. \vspc\\
\end{tabularx}

\section{Projetos Pessoais (Hobbie)}
\begin{tabularx}{1\linewidth}{X}
{\bf Live Wallpaper para Android \hfill Iniciado em Set/2013} \\
Papel de parede dinâmico para a plataforma Android. Foi desenvolvido em Java e API Android. Realiza a integração com webservices da empresa SoundCloud. Este projeto é open source e pode ser acessado na minha conta do Github: http://github.com/suelengc. Também está disponível para download na {\href{https://play.google.com/store/apps/details?id=br.com.suelengc.wallpaper}{Google Play}}. \vspc\\
\end{tabularx}

\begin{tabularx}{1\linewidth}{X}
{\bf Calculadora de Tributos PJ para Android \hfill Iniciado em Jan/2012} \\
Aplicativo para cálculo de tributos, PIS, COFINS, ISS, CSLL e IRPJ, que incidem sobre uma nota fiscal de prestação de serviços no Brasil. Este projeto é Open Source e pode ser acessado na minha conta do Github: http://github.com/suelengc. Também está disponível para download na {\href{https://play.google.com/store/apps/details?id=br.com.suelengc.calctributospj}{Google Play}} \vspc\\
\end{tabularx}

%\section{Conhecimentos}
%\begin{tabularx}{1\linewidth}{p{3.5cm}X}
%Práticas Ágeis:  & Scrum, Lean, Kanban, XP, TDD, DDD;\\
%Boas Práticas:   & Design Patterns, Object Oriented, Clean Code;\\
%Linguagens:      & LaTeX, Java, JSP, Servlet, JSTL, EL, C\#.Net, ASP.Net, VB.Net, ASP, PHP, C/C++;\\
%Web:             & HTML, CSS, XML, XSL, DTD, Javascript, Json, Ajax and Transact SQL;\\
%Mobile:          & Android, Fragments, Robolectric, SQLite and others;\\
%Frameworks:      & .Net until 4.5, Spring, Hibernate, JQuery, Twitter Bootstrap;\\
%Banco de Dados:  & MySQL, Oracle and Microsoft SQL Server; \\
%Servidores:      & Websphere, IIS, Tomcat and Apache;\\
%IDEs:            & Eclipse, MS Visual Studio, Netbeans, Android Studio;\\
%Ferramentas:     & Maven, Jenkins, Hudson;\\
%Versionamento:   & Github, Bitbucket, Source Safe, TFS;\\
%\end{tabularx}

\end{center}

\end{document}