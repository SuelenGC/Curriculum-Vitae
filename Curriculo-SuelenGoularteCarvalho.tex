\documentclass[a4paper, oneside, final]{scrartcl}

\usepackage[utf8]{inputenc}
\usepackage[brazil]{babel}
\usepackage{amssymb}

\usepackage{soul}
\usepackage{scrpage2}
\usepackage{titlesec}
\usepackage{marvosym}
\usepackage{tabularx}
\usepackage{textcomp}

\usepackage[hmargin=2cm,vmargin=2.5cm,noheadfoot]{geometry}

\titleformat{\section}{\large\scshape\raggedright}{}{0em}{}[\titlerule]
\pagestyle{scrheadings}
\renewcommand{\headfont}{\normalfont\rmfamily\scshape}

% add the symbols for email and phone contact data
\cofoot{
\so{ {\Large\Letter} suelengcarvalho@gmail.com \ {\Large\Telefon} +55 (11) 99336-7069 }
%} 
}

\begin{document}

\begin{center}
\textsc{\Huge{Suelen Goularte Carvalho}}\\
\ \\
Brasileira, +55 11 99336-7069, suelengcarvalho@gmail.com

%\textsc{\huge{\so{Suelen Goularte Carvalho}}}\\ \ \\
%{\small \tt Preten��o Salarial: R\$ 10.000,00 CLT Full}

\section{Objetivo}
	Atuar na área de tecnologia da informação como arquiteto de software, especialista ou líder técnico.

%\section{Personal Information}

%\begin{tabularx}{.97\linewidth}{p{3.0cm}p{4.0cm}p{3.0cm}X}
%Phone:           & +55 (11) 9336-7069 & Email:        & suelengcarvalho@gmail.com \\
%Date of Birth: & 26/02/1987     & Marital Status: & Single\\
%\                   & \              & Linkedin:     & %http://www.linkedin.com/in/suelengc 
%\end{tabularx}

\section{Resumo}

\begin{tabularx}{0.97\linewidth}{X}
	Arquiteta de software com mais de 8 anos de experiência em programação com grande entusiasmo por desenvolvimento e novas tecnologias. \\ \ \\
	
	Excelentes conhecimentos em diferentes linguagens de programação e tecnologias, incluindo Java, .Net, Scala, Ruby, C/C++, Android, tecnologias web e SQL.
\end{tabularx}

\section{Formação Acadêmica}

\begin{tabularx}{0.97\linewidth}{p{2cm}X}
2010$-$2012 & Pós-Graduação em Tecnologia da Informação\\
            & Instituto Tecnológico de Aeronáutica, ITA\\
            & Título: Padrões Para Implantar Métodos Ágeis\\            
            & Área: Gestão Estratégica de Projetos\\            
            & \\
\end{tabularx}
\begin{tabularx}{0.97\linewidth}{p{2cm}X}
2005$-$2008 & Graduação em Análise e Desenvolvimento de Sistemas\\
            & Faculdade de Tecnologia Diamantes, FATED\\
            & \\
\end{tabularx}
\begin{tabularx}{0.97\linewidth}{p{2cm}X}
2003$-$2003 & Técnico em Processamento de Dados\\
            & Escola Profissional Nossa Senhora de Fátima, SENAI\\
\end{tabularx}

\section{Formação Complementar}

\begin{tabularx}{0.97\linewidth}{p{2cm}X}
2013       & Jenkins: Automação de deploy e Integração Contínua (12h)\\
           & Caelum\\ {\tiny }
           & \\
\end{tabularx}
\begin{tabularx}{0.97\linewidth}{p{2cm}X}
2013       & Maven: Gerenciando de dependências (12h)\\
           & Caelum\\ 
           & \\
\end{tabularx}
\begin{tabularx}{0.97\linewidth}{p{2cm}X}
2013       & Testes de Unidade e TDD (16h)\\
           & Caelum\\ {\tiny }
           & \\
\end{tabularx}
\begin{tabularx}{0.97\linewidth}{p{2cm}X}
2013       & Java Web (40h)\\
           & Caelum\\ {\tiny }
           & \\
\end{tabularx}
\begin{tabularx}{0.97\linewidth}{p{2cm}X}
           
2012       & Gerenciamento ágil de projetos de Software com Scrum (20h)\\
           & Caelum\\ 
           & \\
\end{tabularx}

\begin{tabularx}{0.97\linewidth}{p{2cm}X}
2012       & Otimizando o Fluxo de Trabalho (8h)\\
           & Virada Ágil - Thought Works\\ 
           & \\
\end{tabularx}

\begin{tabularx}{0.97\linewidth}{p{2cm}X}
2012       & Google Maps Engine (20h)\\
           & Google\\ 
           & \\
\end{tabularx}

\begin{tabularx}{0.97\linewidth}{p{2cm}X}
2012       & Google Maps API (20h)\\
           & Google\\ 
           & \\
\end{tabularx}

\begin{tabularx}{0.97\linewidth}{p{2cm}X}
2012       & Google Android Hands On (8h)\\
           & Tempo Real Eventos\\ 
           & \\
\end{tabularx}

\begin{tabularx}{0.97\linewidth}{p{2cm}X}
2010       & Academia Java (128h)\\
           & Globalcode\\ 
           & \\
\end{tabularx}

\begin{tabularx}{0.97\linewidth}{p{2cm}X}
2010       & Álgebra Linear (120h) \\
           & Universidade de São Paulo, USP\\
           & \\
\end{tabularx}

\begin{tabularx}{0.97\linewidth}{p{2cm}X}
2009       & Tópicos de Programação (60h)\\
           & Universidade de São Paulo, USP\\
           & \\
\end{tabularx}

\begin{tabularx}{0.97\linewidth}{p{2cm}X}
2008       & Academia .Net (VB, ASP, C\# e ADO) (144h)\\
           & Ká Solutions\\
           & \\
\end{tabularx}

\begin{tabularx}{0.97\linewidth}{p{2cm}X}
2004       & Inglês (240h)\\
           & Escola Profissional Nossa Senhora de Fátima, SENAI
\end{tabularx}

\section{Idiomas}

\begin{tabularx}{0.97\linewidth}{p{2cm}X}
Português      	& Nativo\\
Inglês      	& Avançado\\
\end{tabularx}

\section{Atuação Profissional}

\begin{tabularx}{0.97\linewidth}{p{2cm}X}
Empresa     & MapLink \\
Cargo       & Arquiteto de Software \\
Período     & 08/2012 $-$ Atual \\
Atividades  & Atuando diretamente na definição de arquitetura e design de web services e bibliotecas mobile Android e iOS do MapLink. Principal responsável pela evolução das bibliotecas mobile e conteúdo do site dev.maplink.com.br/mobile-api. Atuando diretamente no desenvolvimento da biblioteca Android. Aplicação de integração contínua com Jenkins e Maven, execução automatizada de testes e code review. Estas bibliotecas visam facilitar o uso de serviços web do MapLink em aplicativos móveis.\\ \ \\
\end{tabularx}
\begin{tabularx}{0.97\linewidth}{p{2cm}X}
Empresa     & Liberty Seguros $-$ Emphasys Consultoria \\
Cargo       & Líder de Projetos Web \\
Período     & 04/2011 $-$ 07/2012 \\
Atividades  & Principal responsável pela evolução do Portal do Segurado, liderando os projetos desde a análise até a implantação. Os desafios incluiam questões de performance, escalabilidade, segurança da informação e agilidade na entrega dos projetos. Foram utilizadas práticas ágeis sempre considerando o contexto da empresa. O Portal do Segurado é principal software disponibilizado aos segurados Liberty. \\ \ \\
\end{tabularx}
\begin{tabularx}{0.97\linewidth}{p{2cm}X}
Empresa     & Serasa Experian\\
Cargo       & Analista de Sistemas\\
Período     & 10/2010 $-$ 04/2011\\
Atividades  & Atuando diretamente na implantação de práticas ágeis. Forte atuação na evolução de softwares desenvolvidos em JavaEE e C\#.Net.\\ \ \\
\end{tabularx}
\begin{tabularx}{0.97\linewidth}{p{2cm}X}
Empresa     & Liberty Seguros $-$ Stefanini Consultoria \\
Cargo       & Analista de Sistemas Sênior \\
Período     & 11/06 $-$ 10/2010 \\
Atividades  & Responsável direta desde o desenvolvimento até a implantação de evoluções e correções no sistema Affinity Web. O Affinity Web é o principal software de cotação da Liberty. As tecnologias utilizadas foram VB.Net, ASP.NET e Javascript. A arquitetura incluia o conceito de multiplas camadas e software orientado a serviços (SOA) utilizando o protocolo Soap e XML como formato de troca de dados.\\ \ \\
\end{tabularx}
\begin{tabularx}{0.97\linewidth}{p{2cm}X}
Empresa     & Intersaúde $-$ Emphasys Consultoria \\
Cargo       & Analista Programador Pleno \\
Período     & 09/2006 $-$ 11/2006  \\
Atividades  & Atuação no desenvolvimento com Visual Basic e Delphi. \\ \ \\
\end{tabularx}
\begin{tabularx}{0.97\linewidth}{p{2cm}X}
Empresa     & Signa Logística do E-Business \\
Cargo       & Programador Pleno \\
Período     & 04/2006 $-$ 09/2006  \\
Atividades  & Desenvolvimento de sistemas utilizando multiplas tecnologias $-$ ASP, VBScript, Visual Basic, HTML, XML, CSS e JavaScript. Desenvolvimento e manutenção de stored procedures, triggers em sistemas Oracle e Microsoft SQL Server. \\ \ \\ 
\end{tabularx}

\section{Atuação Acadêmica}
\begin{tabularx}{0.97\linewidth}{p{2cm}X}
Empresa     & Caelum \\
Cargo       & Instrutor \\
Período     & 10/2012 $-$ Atual \\
Atividades  & Treinamentos dos cursos de Java para Web, Gerenciamento Ágil de Projetos de Software com Scrum, Desenvolvimento Móvel com Google Android e Técnicas de Desenvolvimento Android avançado. \\ \ \\
\end{tabularx}
\begin{tabularx}{0.97\linewidth}{p{2cm}X}
Empresa     & Universidade Bandeirantes, UNIBAN \\
Cargo       & Professor \\
Período     & 04/2009 $-$ 08/2009 \\
Atividades  & Professor da disciplina de graduação Desenvolvimento Web. \\ \ \\
\end{tabularx}
\begin{tabularx}{0.97\linewidth}{p{2cm}X}
Empresa     & Faculdade de Tecnologia Diamantes, FATED \\
Cargo       & Professor  \\
Período     & 06/2008 $-$ 06/2009 \\
Atividades  & Professor das disciplinas de graduação Desenvolvimento Web e Práticas de Orientação a Objeto com Java. \\ \ \\
\end{tabularx}
\begin{tabularx}{0.97\linewidth}{p{2cm}X}
Empresa     & Faculdade Paulista de Artes, FPA \\
Cargo       & Professor \\
Período     & 05/2007 $-$ 07/2007 \\
Atividades  & Professor da disciplina do curso extracurricular Desenvolvimento Web com PHP. \\ \ \\
\end{tabularx}

\section{Participações em Eventos (Como Palestrante)}
\begin{tabularx}{0.97\linewidth}{p{2cm}X}
08-2013    & QConSP\\
           & Apresentação Nativo ou Cross-Plataform: Caminhos e Alternativas\\
           & \\
\end{tabularx}

\begin{tabularx}{0.97\linewidth}{p{2cm}X}
07-2013    & 5º GDG Android Meetup\\
           & Apresentação sobre O bê-a-bá de Widgets\\
           & \\
\end{tabularx}

\begin{tabularx}{0.97\linewidth}{p{2cm}X}
07-2013    & 3º GDG Android Meetup\\
           & Apresentação Usando o Poder da API Fragments\\
           & \\
\end{tabularx}

\begin{tabularx}{0.97\linewidth}{p{2cm}X}
04-2013    & DevFestW\\
           & Apresentação Primeiros Passos com Android\\
           & \\
\end{tabularx}

\begin{tabularx}{0.97\linewidth}{p{2cm}X}
11-2012    & DevFest\\
           & Apresentação Esquartejando sua Activity com Fragments\\
           & \\
\end{tabularx}

\begin{tabularx}{0.97\linewidth}{p{2cm}X}
09-2012    & Agile Brazil\\
           & Apresentação Padrões Para Implantar Métodos Ágeis\\
           & \\
\end{tabularx}

\begin{tabularx}{0.97\linewidth}{p{2cm}X}
08-2012    & QConSP\\
           & Apresentação SOLID em 5 minutos\\
           & \\
\end{tabularx}

\begin{tabularx}{0.97\linewidth}{p{2cm}X}
07-2012    & TDC - The Developer's Conference\\
           & Apresentação 7 Padrões Para Implantar Métodos Ágeis\\
           & \\
\end{tabularx}

\begin{tabularx}{0.97\linewidth}{p{2cm}X}
07-2012    & TDC - The Developer's Conference\\
           & Apresentação Deixando sua Interface mais Bonita com Shapes\\
           & \\
\end{tabularx}

\begin{tabularx}{0.97\linewidth}{p{2cm}X}
09-2011    & MiniPLoP Brasil\\
           & Artigo Padrões Para Implantar Métodos Ágeis\\
           & \\
\end{tabularx}

\section{Conhecimentos}

\begin{tabularx}{0.97\linewidth}{p{3.5cm}X}
Práticas Ágeis: 	& Scrum, Lean, Kanban, XP, TDD e DDD.\\
Práticas:  			& Design Patterns, Orientação a Objetos.\\
Linguagens:       	& Java, JSP, Servlet, JSTL, EL, C\#.Net, ASP.Net, VB.Net, ASP, PHP, C/C++.\\
Tecnologias:    	& HTML, CSS, XML, XSL, DTD, Javascript, Json, Ajax, LaTeX e SQL.\\
Mobile:          	& Android, ActionBarSherlock, KSoap2, Fragments, Robotium, HttpSync, AsyncTask, SQLite.\\
Frameworks:      	& .Net até 4.5, Spring, Hibernate, JQuery.\\
Banco de Dados:     & MySQL, Oracle and Microsoft SQL Server \\
Servidores:         & Websphere, IIS, Tomcat e Apache.\\
IDE's:            	& Eclipse, MS Visual Studio, Netbeans.\\
Ferramentas: 		& Maven, Jenkins, Hudson.\\
Controle de Versão: & Github, Bitbucket, Source Safe, TFS.\\
Aplicativos:    	& Photoshop, Ilustrator and Dreamweaver.\\
\end{tabularx}

\end{center}

\end{document}