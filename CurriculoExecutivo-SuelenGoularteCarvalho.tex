\documentclass[a4paper, oneside, final]{scrartcl}

\usepackage[utf8]{inputenc}
\usepackage[brazil]{babel}
\usepackage{amssymb}

\usepackage{soul}
\usepackage{scrpage2}
\usepackage{titlesec}
\usepackage{marvosym}
\usepackage{tabularx}
\usepackage{textcomp}
\newcommand{\vspc}{\vspace{0.15cm}} %espaçamento vertical
\newcommand{\vspcitem}{\vspace{0.1cm}} %espaçamento vertical

\usepackage[hmargin=1.0cm,vmargin=1.0cm,noheadfoot]{geometry}

\titleformat{\section}{\large\scshape\raggedright}{}{0em}{}[\titlerule]
\pagestyle{scrheadings}
\renewcommand{\headfont}{\normalfont\rmfamily\scshape}

% add the symbols for email and phone contact data
\cofoot{
%\so{ {\Large\Letter} suelengcarvalho@gmail.com \ {\Large\Telefon} +55 (11) 99336-7069 }
%} 
}

\begin{document}

\begin{center}
\textsc{\Huge{Suelen Goularte Carvalho}} \vspc\\
{\small Brasileira, +55 11 99336-7069, suelengcarvalho@gmail.com} 

%\textsc{\huge{\so{Suelen Goularte Carvalho}}}\\ \ \\
%{\small \tt Preten��o Salarial: R\$ 10.000,00 CLT Full}

\section{Objetivo}
	{\large \bf Posição Executiva na área de Tecnologia da Informação.}

%\section{Personal Information}

%\begin{tabularx}{.97\linewidth}{p{3.0cm}p{4.0cm}p{3.0cm}X}
%Phone:           & +55 (11) 9336-7069 & Email:        & suelengcarvalho@gmail.com \\
%Date of Birth: & 26/02/1987     & Marital Status: & Single\\
%\                   & \              & Linkedin:     & %http://www.linkedin.com/in/suelengc 
%\end{tabularx}

\section{Resumo}
\begin{tabularx}{1\linewidth}{X}
	$\star$ Carreira em desenvolvimento na área de Tecnologia da Informação, tendo atuado em empresas multinacionais e nacionais de pequeno a grande porte. \vspc\\

	$\star$ Arquiteta de software Web e Mobile com mais de 8 anos de experiência em programação e com grande entusiasmo por software e novas tecnologias. \vspc\\
		
	$\star$ Liderança ágil de times próprios e terceirizados, com foco no desenvolvimento de equipes de alta performance e cumprimento de objetivos estratégicos. \vspc\\
	
	$\star$ Excelentes conhecimentos em diferentes linguagens de programação e tecnologias, incluindo Java, .Net, Scala, C/C++, Android, tecnologias web e SQL. \vspc\\
\end{tabularx}


\section{Idiomas}
\begin{tabularx}{1\linewidth}{p{2cm}X}
Português      	& Nativo\\
Inglês      	& Avançado\\
\end{tabularx}

\section{Formação Acadêmica}
\begin{tabularx}{1\linewidth}{p{2cm}X}
2010$-$2012 & Pós-Graduação em Gestão Estratégica de Projetos\\
            & Instituto Tecnológico de Aeronáutica, ITA\\
            & Título: Padrões Para Implantar Métodos Ágeis\\
            & Área: Tecnologia da Informação \vspc\\
\end{tabularx}
\begin{tabularx}{1\linewidth}{p{2cm}X}
2005$-$2008 & Graduação em Análise e Desenvolvimento de Sistemas\\
            & Faculdade de Tecnologia Diamantes, FATED \vspc\\
\end{tabularx}
\begin{tabularx}{1\linewidth}{p{2cm}X}
2003$-$2003 & Técnico em Processamento de Dados\\
            & Escola Profissional Nossa Senhora de Fátima, SENAI\\
\end{tabularx}


\section{Formação Complementar}
\begin{tabularx}{1\linewidth}{p{2cm}X}
2013       & Jenkins: Automação de deploy e Integração Contínua - Caelum \vspcitem\\
\end{tabularx}

\begin{tabularx}{1\linewidth}{p{2cm}X}
2013       & Maven: Gerenciando de dependências - Caelum \vspcitem\\
\end{tabularx}

\begin{tabularx}{1\linewidth}{p{2cm}X}
2013       & Testes de Unidade e TDD - Caelum \vspcitem\\
\end{tabularx}

\begin{tabularx}{1\linewidth}{p{2cm}X}
2013       & Java Web - Caelum \vspcitem\\
\end{tabularx}

\begin{tabularx}{1\linewidth}{p{2cm}X}
2012       & Gerenciamento ágil de projetos de Software com Scrum - Caelum \vspcitem\\
\end{tabularx}

\begin{tabularx}{1\linewidth}{p{2cm}X}
2012       & Otimizando o Fluxo de Trabalho - Virada Ágil - Thought Works \vspcitem\\
\end{tabularx}

\begin{tabularx}{1\linewidth}{p{2cm}X}
2012       & Google Maps Engine - Google \vspcitem\\
\end{tabularx}

\begin{tabularx}{1\linewidth}{p{2cm}X}
2012       & Google Maps API - Google \vspcitem\\
\end{tabularx}

\begin{tabularx}{1\linewidth}{p{2cm}X}
2012       & Google Android Hands On - Tempo Real Eventos \vspcitem\\
\end{tabularx}

\begin{tabularx}{1\linewidth}{p{2cm}X}
2010       & Academia Java - Globalcode \vspcitem\\
\end{tabularx}

\begin{tabularx}{1\linewidth}{p{2cm}X}
2010       & Álgebra Linear - Universidade de São Paulo, USP \vspcitem\\
\end{tabularx}

\begin{tabularx}{1\linewidth}{p{2cm}X}
2009       & Tópicos de Programação - Universidade de São Paulo, USP \vspcitem\\
\end{tabularx}

\begin{tabularx}{1\linewidth}{p{2cm}X}
2008       & Academia .Net (VB, ASP, C\# e ADO) - Ká Solutions 
\end{tabularx}


\section{Atuação Profissional}
\begin{tabularx}{1\linewidth}{X}
{\bf MapLink $\star$ Arquiteto de Software \hfill 08/2012 $-$ Atual} \\
No primeiro ano atuei na liderança técnica do time de projetos corporativos composto por 8 pessoas. Entregamos dois projetos e criamos diversos serviços que facilitaram trabalhos posteriormente. Atualmente atuo diretamente na definição de arquitetura e design dos web services e bibliotecas mobile Android e iOS do MapLink. Sou a principal responsável pela evolução das APIs mobile e o conteúdo do site dedicado a desenvolvedores (dev.maplink.com.br/mobile-api). \vspc\\
\end{tabularx}

\begin{tabularx}{1\linewidth}{X}
{\bf Liberty Seguros $\star$ Líder de Projetos Web \hfill 04/2011 $-$ 07/2012} \\
Líder de projetos responsável pela evolução do Portal do Segurado, principal software disponibilizado aos segurados Liberty. Atuando com times terceirizados de até 6 pessoas. Foram utilizados conceitos da cultura ágil para a gestão dos projetos. Ao longo de pouco mais de 1 ano foram entregues dois projetos de grande peso estratégico para a companhia. O Portal do Segurado é a principal software disponibilizado aos segurados Liberty.\vspc\\
\end{tabularx}

\begin{tabularx}{1\linewidth}{X}
{\bf Serasa Experian $\star$  Analista de Sistemas \hfill 10/2010 $-$ 04/2011} \\
Lider do time de sustentação composto por 5 pessoas. Foi realizado a implantação de práticas ágeis tais como Lean, Kanban e Scrum e criado parcerias com áreas próximas ao nosso dia-a-dia.\vspc\\
\end{tabularx}

\begin{tabularx}{1\linewidth}{X}
{\bf Liberty Seguros $\star$ Analista de Sistemas Sênior \hfill 11/2006 $-$ 10/2010} \\
Responsável direta desde o desenvolvimento até a implantação de evoluções e correções no sistema Affinity Web. Arquitetura baseada no conceito de multiplas camadas e SOA utilizando o protocolo Soap e XML.\vspc\\
\end{tabularx}

\begin{tabularx}{1\linewidth}{X}
{\bf Emphasys Consultoria $\star$ Analista Programador Pleno  \hfill 09/2006 $-$ 11/2006} \\
Atuação no desenvolvimento com Visual Basic e Delphi. \vspc\\
\end{tabularx}

\begin{tabularx}{1\linewidth}{X}
{\bf Signa Logística do E-Business $\star$ Programador Pleno  \hfill 04/2006 $-$ 09/2006} \\
Desenvolvimento de sistemas utilizando multiplas tecnologias $-$ ASP, VBScript, Visual Basic, HTML, XML, CSS e JavaScript. Desenvolvimento e manutenção de stored procedures, triggers em sistemas Oracle e Microsoft SQL Server. \vspc\\
\end{tabularx}

\section{Atuação Acadêmica}
\begin{tabularx}{1\linewidth}{X}
{\bf Caelum $\star$ Instrutor  \hfill 10/2012 $-$ Atual} \\
Treinamentos dos cursos de Java para Web, Gerenciamento Ágil de Projetos de Software com Scrum, Desenvolvimento Móvel com Google Android e Técnicas de Desenvolvimento Android avançado. \vspc\\
\end{tabularx}

\begin{tabularx}{1\linewidth}{X}
{\bf Universidade Bandeirantes, UNIBAN $\star$ Professor \hfill 04/2009 $-$ 08/2009} \\
Professor da disciplina de graduação Desenvolvimento Web. \vspc\\
\end{tabularx}

\begin{tabularx}{1\linewidth}{X}
{\bf Faculdade de Tecnologia Diamantes, FATED $\star$ Professor \hfill 06/2008 $-$ 06/2009} \\
Professor das disciplinas de graduação Desenvolvimento Web e Práticas de Orientação a Objeto com Java. \vspc\\
\end{tabularx}

\begin{tabularx}{1\linewidth}{X}
{\bf Faculdade Paulista de Artes, FPA $\star$ Professor  \hfill 05/2007 $-$ 07/2007} \\
Professor da disciplina do curso extracurricular Desenvolvimento Web com PHP. \vspc\\
\end{tabularx}


\section{Participações em Eventos (Como Palestrante)}
\begin{tabularx}{1\linewidth}{p{2.0cm}X}
08/2013    & QConSP - Nativo ou Cross-Plataform: Caminhos e Alternativas \vspcitem\\
\end{tabularx}
\begin{tabularx}{1\linewidth}{p{2.0cm}X}
07/2013    & 5º GDG Android Meetup - sobre O bê-a-bá de Widgets \vspcitem\\
\end{tabularx}
\begin{tabularx}{1\linewidth}{p{2.0cm}X}
07/2013    & 3º GDG Android Meetup - Usando o Poder da API Fragments \vspcitem\\
\end{tabularx}
\begin{tabularx}{1\linewidth}{p{2.0cm}X}
04/2013    & DevFestW - Primeiros Passos com Android \vspcitem\\
\end{tabularx}
\begin{tabularx}{1\linewidth}{p{2.0cm}X}
11/2012    & DevFest - Esquartejando sua Activity com Fragments \vspcitem\\
\end{tabularx}
\begin{tabularx}{1\linewidth}{p{2.0cm}X}
09/2012    & Agile Brazil - Padrões Para Implantar Métodos Ágeis \vspcitem\\
\end{tabularx}
\begin{tabularx}{1\linewidth}{p{2.0cm}X}
08/2012    & QConSP - SOLID em 5 minutos \vspcitem\\
\end{tabularx}
\begin{tabularx}{1\linewidth}{p{2.0cm}X}
07/2012    & TDC - The Developer's Conference - 7 Padrões Para Implantar Métodos Ágeis \vspcitem\\
\end{tabularx}
\begin{tabularx}{1\linewidth}{p{2.0cm}X}
07/2012    & TDC - The Developer's Conference - Deixando sua Interface mais Bonita com Shapes \vspcitem\\
\end{tabularx}
\begin{tabularx}{1\linewidth}{p{2.0cm}X}
09/2011    & MiniPLoP Brasil - Artigo Padrões Para Implantar Métodos Ágeis 
\end{tabularx}

%\section{Conhecimentos}
%\begin{tabularx}{1\linewidth}{p{3.5cm}X}
%Práticas Ágeis: 	& Scrum, Lean, Kanban, XP, TDD e DDD.\\
%Práticas:  			& Design Patterns, Orientação a Objetos.\\
%Linguagens:       	& Java, JSP, Servlet, JSTL, EL, C\#.Net, ASP.Net, VB.Net, ASP, PHP, C/C++.\\
%Tecnologias:    	& HTML, CSS, XML, XSL, DTD, Javascript, Json, Ajax, LaTeX e SQL.\\
%Mobile:          	& Android, ActionBarSherlock, KSoap2, Fragments, Robotium, HttpSync, AsyncTask, SQLite.\\
%Frameworks:      	& .Net até 4.5, Spring, Hibernate, JQuery.\\
%Banco de Dados:     & MySQL, Oracle and Microsoft SQL Server \\
%Servidores:         & Websphere, IIS, Tomcat e Apache.\\
%IDE's:            	& Eclipse, MS Visual Studio, Netbeans.\\
%Ferramentas: 		& Maven, Jenkins, Hudson.\\
%Controle de Versão: & Github, Bitbucket, Source Safe, TFS.\\
%Aplicativos:    	& Photoshop, Ilustrator and Dreamweaver.\\
%\end{tabularx}

\end{center}

\end{document}