\documentclass[a4paper, oneside, final]{article}

\usepackage[utf8]{inputenc}
\usepackage[brazil]{babel}
\usepackage{amssymb}

\usepackage{soul}
\usepackage{scrpage2}
\usepackage{titlesec}
\usepackage{marvosym}
\usepackage{tabularx}
\usepackage{textcomp}

\usepackage[hmargin=2cm,vmargin=2.5cm,noheadfoot]{geometry}

\titleformat{\section}{\large\scshape\raggedright}{}{0em}{}[\titlerule]
\pagestyle{scrheadings}
\renewcommand{\headfont}{\normalfont\rmfamily\scshape}

% add the symbols for email and phone contact data
\cofoot{
\so{ {\Large\Letter} suelengcarvalho@gmail.com \ {\Large\Telefon} +55 (11) 99336-7069 }
%} 
}

\begin{document}

\begin{center}
\textsc{\Huge{Suelen Goularte Carvalho}}\\
\ \\
Brasileira, +55 11 99336-7069, suelengcarvalho@gmail.com

%\textsc{\huge{\so{Suelen Goularte Carvalho}}}\\ \ \\
%{\small \tt Preten��o Salarial: R\$ 10.000,00 CLT Full}

\section{Objetivo}
	ARQUITETA DE SOFTWARE / DESENVOLVEDOR DE SOFTWARE SENIOR

%\section{Personal Information}

%\begin{tabularx}{.97\linewidth}{p{3.0cm}p{4.0cm}p{3.0cm}X}
%Phone:           & +55 (11) 9336-7069 & Email:        & suelengcarvalho@gmail.com \\
%Date of Birth: & 26/02/1987     & Marital Status: & Single\\
%\                   & \              & Linkedin:     & %http://www.linkedin.com/in/suelengc 
%\end{tabularx}

\section{Qualificações}

\begin{itemize}
	\item Palestrante em eventos de tecnologia e métodos ágeis.
	\item Três anos de atuação como líder de projetos e time.
	\item Mais de 8 anos atuando com desenvolvimento de softwares web e mobile nas plataformas Java, .Net, Android.
	\item Três anos lecionando sobre desenvolvimento de softwares, boas práticas e técnicas. 
	\item Últimos três anos trabalhando diariamente com métodos ágeis como Scrum, Lean, Kanban, TDD e outros. 
	\item Comunicativa. Responsiva. Altamente empreendedora. Com disposição para enfrentar desafios e projetos inovadores.\\
\end{itemize}

\section{Atuação Profissional}

\begin{tabularx}{0.97\linewidth}{p{2cm}X}
Empresa     & MapLink \\
Cargo       & Arquiteto de Software \\
Período     & 08/2012 $-$ Atual \\
Atividades  & Atuando na definição e otimização da arquitetura de softwares web de geomarketing e API de Mapas para aplicações Android. Atuo também na implantação de Agile para desenvolvimento das tarefas dos projetos e dia-a-dia. \\
            & \ \\
\end{tabularx}
\begin{tabularx}{0.97\linewidth}{p{2cm}X}
Empresa     & Liberty Seguros $-$ Emphasys Consultoria \\
Cargo       & Líder de Projetos Web \\
Período     & 04/2011 $-$ 07/2012 \\
Atividades  & Atuando na liderança de projetos web que utilizam as tecnologias VB.Net, ASP.Net (framework do 1.1 até o 4.0 e Visual Studio do 2003 ao 2010) e SharePoint 10. Minhas tarefas incluem análise e desenvolvimento, análise de viabilidade tecnológica, elaboração de cronograma, definições de padrões e arquitetura utilizadas no desenvolvimento, elaboração de diagramas de comunicação, classes, dentre outros, gerenciamento de expectativas e mudanças. \\
            & \ \\
\end{tabularx}
\begin{tabularx}{0.97\linewidth}{p{2cm}X}
Empresa     & Serasa Experian\\
Cargo       & Analista de Sistemas\\
Período     & 10/2010 $-$ 04/2011\\
Atividades  & Implantação da gestão ágil na equipe de manutenção corretiva. Desenvolvimento Java Web (utilizando Eclipse, Websphere e Harvest). Desenvolvimento VB.Net e C\#.Net (framework 1.1 até 4.0 com Visual Studio do 2003 até 2010). Implementação de métricas para melhoria da qualidade de desenvolvimento de software. \\
            & \ \\
\end{tabularx}
\begin{tabularx}{0.97\linewidth}{p{2cm}X}
Empresa     & Liberty Seguros $-$ Stefanini Consultoria \\
Cargo       & Analista de Sistemas Sênior \\
Período     & 11/06 $-$ 10/2010 \\
Atividades  & Desenvolvimento e manutenção de sistemas e webservices em VB.Net, ASP.NET e Javascript utilizando arquitetura de multiplas camadas e SOA. As principais atividades incluiram desenvolvimento Web, elaboração de planejamento e cronogramas, negociação de prazos, definições de diagramas junto a área de arquitetura, acompanhamento de homologações, suporte aos usuários e delegação de atividades para outros analistas. \\
            & \ \\
\end{tabularx}
\begin{tabularx}{0.97\linewidth}{p{2cm}X}
Empresa     & Intersaúde $-$ Emphasys Consultoria \\
Cargo       & Analista Programador Pleno \\
Período     & 09/2006 $-$ 11/2006  \\
Atividades  & Participação em projetos de desenvolvimento utilizando Visual Basic e Delphi. \\ 
            & \ \\
\end{tabularx}
\begin{tabularx}{0.97\linewidth}{p{2cm}X}
Empresa     & Signa Logística do E-Business \\
Cargo       & Programador Pleno \\
Período     & 04/2006 $-$ 09/2006  \\
Atividades  & Desenvolvimento de sistemas utilizando multiplas tecnologias $-$ ASP, VBScript, Visual Basic, HTML, XML, CSS e JavaScript. Desenvolvimento e manutenção de stored procedures, triggers em sistemas Oracle e Microsoft SQL Server. \ \\ 
\end{tabularx}

\section{Atuação Acadêmica}
\begin{tabularx}{0.97\linewidth}{p{2cm}X}
Empresa     & Caelum \\
Cargo       & Instrutor \\
Período     & 10/2012 $-$ Atual \\
Atividades  & Treinamentos dos cursos de Java para Web, Gerenciamento Ágil de Projetos de Software com Scrum e Desenvolvimento Móvel com Google Android. \\
            & \ \\
\end{tabularx}
\begin{tabularx}{0.97\linewidth}{p{2cm}X}
Empresa     & Universidade Bandeirantes, UNIBAN \\
Cargo       & Professor \\
Período     & 04/2009 $-$ 08/2009 \\
Atividades  & Professor da disciplina de graduação Desenvolvimento Web. \\ 
            & \ \\
\end{tabularx}
\begin{tabularx}{0.97\linewidth}{p{2cm}X}
Empresa     & Faculdade de Tecnologia Diamantes, FATED \\
Cargo       & Professor  \\
Período     & 06/2008 $-$ 06/2009 \\
Atividades  & Professor das disciplinas de graduação Desenvolvimento Web e Práticas de Orientação a Objeto com Java. \\ 
            & \ \\
\end{tabularx}
\begin{tabularx}{0.97\linewidth}{p{2cm}X}
Empresa     & Faculdade Paulista de Artes, FPA \\
Cargo       & Professor \\
Período     & 05/2007 $-$ 07/2007 \\
Atividades  & Professor da disciplina do curso extracurricular Desenvolvimento Web com PHP. \\ 
\end{tabularx}

\section{Formação Acadêmica}

\begin{tabularx}{0.97\linewidth}{p{2cm}X}
2010$-$2012 & Pós-Graduação em Gestão Estratégica de Projetos\\
            & Instituto Tecnológico de Aeronáutica, ITA\\
            & \\
\end{tabularx}
\begin{tabularx}{0.97\linewidth}{p{2cm}X}
2005$-$2008 & Graduação em Análise e Desenvolvimento de Sistemas\\
            & Faculdade de Tecnologia Diamantes, FATED\\
            & \\
\end{tabularx}
\begin{tabularx}{0.97\linewidth}{p{2cm}X}
2003$-$2003 & Técnico em Processamento de Dados\\
            & Escola Profissional Nossa Senhora de Fátima, SENAI\\
\end{tabularx}

\section{Formação Complementar}

\begin{tabularx}{0.97\linewidth}{p{2cm}X}
2013       & Java Web (40h)\\
           & Caelum\\ {\tiny }
           & \\
\end{tabularx}
\begin{tabularx}{0.97\linewidth}{p{2cm}X}
           
2012       & Gerenciamento ágil de projetos de Software com Scrum (20h)\\
           & Caelum\\ 
           & \\
\end{tabularx}

\begin{tabularx}{0.97\linewidth}{p{2cm}X}
2012       & Otimizando o Fluxo de Trabalho (8h)\\
           & Virada Ágil - Thought Works\\ 
           & \\
\end{tabularx}

\begin{tabularx}{0.97\linewidth}{p{2cm}X}
2012       & Google Maps Engine (20h)\\
           & Google\\ 
           & \\
\end{tabularx}

\begin{tabularx}{0.97\linewidth}{p{2cm}X}
2012       & Google Maps API (20h)\\
           & Google\\ 
           & \\
\end{tabularx}

\begin{tabularx}{0.97\linewidth}{p{2cm}X}
2012       & Google Android Hands On (8h)\\
           & Tempo Real Eventos\\ 
           & \\
\end{tabularx}

\begin{tabularx}{0.97\linewidth}{p{2cm}X}
2010       & Academia Java (128h)\\
           & Globalcode\\ 
           & \\
\end{tabularx}

\begin{tabularx}{0.97\linewidth}{p{2cm}X}
2010       & Álgebra Linear (120h) \\
           & Universidade de São Paulo, USP\\
           & \\
\end{tabularx}

\begin{tabularx}{0.97\linewidth}{p{2cm}X}
2009       & Tópicos de Programação (60h)\\
           & Universidade de São Paulo, USP\\
           & \\
\end{tabularx}

\begin{tabularx}{0.97\linewidth}{p{2cm}X}
2008       & Academia .Net (VB, ASP, C\# e ADO) (144h)\\
           & Ká Solutions\\
           & \\
\end{tabularx}

\begin{tabularx}{0.97\linewidth}{p{2cm}X}
2004       & Inglês (240h)\\
           & Escola Profissional Nossa Senhora de Fátima, SENAI
\end{tabularx}

%\section{Certifica��es}
%
%\begin{tabularx}{0.97\linewidth}{p{2cm}X}
%2011        & SCJP --- Sun Certified Java Programmer\\ \\
%\end{tabularx}

\section{Idiomas}

\begin{tabularx}{0.97\linewidth}{p{2cm}X}
Protuguês      & Nativo\\
Inglês      & Avançado\\
\end{tabularx}

\section{Conhecimentos}

\begin{tabularx}{0.97\linewidth}{p{3.0cm}X}
Práticas Ágeis: 	& Scrum, Lean, Kanban, XP, TDD, DDD.\\
Práticas:  			& Design Patterns, Object Oriented.\\
Linguagens:       	& Java, JSP, Servlet, JSTL, EL, C\#.Net, ASP.Net, VB.Net, ASP, PHP, C/C++.\\
Tecnologias:    	& HTML, CSS, XML, XSL, DTD, Javascript, Json, Ajax and Transact SQL.\\
Mobile:          	& Android, ActionBarSherlock, KSoap2, API Fragments, Robotium, HttpSync, AsyncTask. AR Mixare, SQLite.\\
Frameworks:      	& .Net until 4.5, Spring, Hibernate, JQuery.\\
Banco de Dados:     & MySQL, Oracle and Microsoft SQL Server \\
Servidores:         & Websphere, IIS, Tomcat and Apache.\\
IDE's:            	& Eclipse, MS Visual Studio, Netbeans.\\
Controle de Versão: & Github, Bitbucket, Source Safe, TFS.\\
Aplicativos:    	& Photoshop, Ilustrator and Dreamweaver.\\
\end{tabularx}

\section{Participações em Eventos (Como Palestrante)}
\begin{itemize}
	\item 2013-07-02 - Apresentação sobre O bê-a-bá de Widgets no 5º GDG Android Meetup
	\item 2013-05-07 - Apresentação Usando o Poder da API Fragments no 3º GDG Android Meetup
	\item 2013-04-03 - Apresentação Primeiros Passos com Android no DevFestW
	\item 2012-11-30 - Apresentação Esquartejando sua Activity com Fragments no DevFest
	\item 2012-09-06 - Apresentação Padrões Para Implantar Métodos Ágeis no Agile Brazil
	\item 2012-08-04 - Apresentação SOLID em 5 minutos no QConSP
	\item 2012-07-08 - Apresentação 7 Padrões Para Implantar Métodos Ágeis no The Developer's Conference
	\item 2012-07-05 - Apresentação Deixando sua Interface mais Bonita com Shapes no The Developer's Conference
	\item 2012-03-09 - Monografia Padrões Para Implantar Métodos Ágeis no ITA
	\item 2011-09-25 - Artigo Padrões Para Implantar Métodos Ágeis no MiniPLoP Brasil
\end{itemize}

%\section{Interests}

\end{center}

\end{document}